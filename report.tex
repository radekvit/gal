\documentclass[12pt, a4paper]{article}
\usepackage[czech]{babel}
\usepackage[utf8]{inputenc}
\usepackage{amsmath}
\usepackage{amsthm}
\usepackage{amssymb}
\usepackage{enumitem}
\usepackage{algorithm}
\usepackage{algpseudocode}
\usepackage{mathtools}
\PassOptionsToPackage{hyphens}{url}\usepackage{hyperref}

\usepackage{calrsfs}

\DeclareMathAlphabet{\pazocal}{OMS}{zplm}{m}{n}

\newcommand{\Lb}{\pazocal{L}}

%define Input and Output commands in algorithmicx
\algnewcommand\algorithmicinput{\textbf{Vstup:}}
\algnewcommand\Input{\item[\algorithmicinput]}

\algnewcommand\algorithmicoutput{\textbf{Výstup:}}
\algnewcommand\Output{\item[\algorithmicoutput]}

\let\oldemptyset\emptyset
\let\emptyset\varnothing
\let\oldepsilon\epsilon
\let\epsilon\varepsilon
\let\emptystring\varepsilon

\def\CC{{C\nolinebreak[4]\hspace{-.05em}\raisebox{.4ex}{\tiny\bf ++}}}
\def\CS{{\settoheight{\dimen0}{C}C\kern-.05em \resizebox{!}{\dimen0}{\raisebox{\depth}{\#}}}}

\author{Martin Dočekal \texttt{xdocek09}\\Radek Vít \texttt{xvitra00}}
\title{
	\includegraphics[scale=0.6]{FIT_barevne_PANTONE_CZ.pdf}\\
	GAL 2018\\
	Porovnání složitosti algoritmů barvení grafů
}


\begin{document}
\newpage
\maketitle

\newpage

\section{Úvod}
Tato práce si dává za cíl vytvořit program, který bude schopen experimentálně porovnávat vybrané algoritmy pro barvení grafů. Bude umožňovat i automatické generování grafů, jelikož manuální tvorba velkých grafů (třeba i jen stovky vrcholů) je problematická.
V druhé části práce představíme zkoumané algoritmy barvení grafů a ukážeme výsledky experimentů.

\section{Barvení grafů}
Barvení grafů je jedním z problémů z teorie grafů. Cílem barvení je přiřazení každému uzlu barvu (číslo) tak, aby žádný sousední uzel neměl stejnou barvu.
Tomuto říkáme barvení vrcholů, tato práce se jím zabývá.
Ovšem existuje například i barvení hran či stěn, kterými se tato práce nezabývá. %todo citace

Nechť $G=(V,E)$ je neorientovaný graf, kde $n = |V|$ a $m = |E|$. Pak obarvením rozumíme funkci:
$$f: 	V \rightarrow B$$
Kde B je množina barev a $\{ u, v \} \in E \Rightarrow f(u) \not= f(v)$. %todo citace
Pro zjednodušení uvažujeme množinu vrcholů velikosti $n$ za $V = \{ i \in \mathbb{N} | 1 \leq i \leq n \}$

\section{Vybrané algoritmy barvení grafů}
V této části představíme implementované a zkoumané algoritmy barvení vrcholů.

Z důvodu velké časové složitosti obarvení grafu nejmenším možným počtem barev používáme
pro barvení grafu heuristiky, které pracují rychle, ale jejich výsledkem je větší než minimální počet barev.
\subsection{Greedy Coloring}
\begin{algorithm}
\caption{Gredy coloring} %todo citace
\label{Greedy coloring}
\begin{algorithmic}
\Input Neorientovaný graf $G = (V, E)$
\Output funkce obarvení grafu $f: 	V \rightarrow B$
\State Vyber náhodnou permutaci $p(1), \dots, p(n)$ čísel $1 \dots |V|$
\ForAll{$i \in <1, |V|>$}
	\State $v \leftarrow p(i)$
	\State $S \leftarrow \{ \text{barvy všech obarvených sousedů uzlu v} \}$
	\State $f(v) \leftarrow \text{Nejmenší barva, která není v S}$ 
\EndFor
\State \Return $f$
\end{algorithmic}
\end{algorithm}

Algoritmus prochází uzly v předem stanoveném pořadí.
V pseudokódu je uváděna náhodná permutace, ale v naší implementaci používáme pořadí vrcholů, které je dáno načtením/vygenerováním daného grafu.
Každý vrchol je obarven nejmenší nepoužitou barvou u svých sousedů.
Z této vlastnosti plyne, že obarvení grafu závisí na zvoleném pořadí uzlů.

Časová složitost algoritmu je $O(n+m)$.
Jelikož procházíme všechny uzly (celkem $O(n)$ iterací)
a při hledání použitelné barvy procházíme všechny sousedy daného uzlu a poznačujeme si použité barvy (celkem $O(m)$).

Nalezení minimální nepoužité barvy lze zvládnout v maximálně $x+1$ krocích (kde $x <= m$).
Protože lze použité barvy ukládat do pole flagů (použitá/nepoužitá) o maximální velikosti $n$ (pro každý vrchol jedna barva),
kde index do pole odpovídá dané barvě.
Potom toto pole od začátku procházet a zastavit se na první nepoužité barvě.
Má-li daný vrchol $x$ sousedů, tak v nejhorším možném případě, kdy sousedi používají všechny barvy $1$ až $x$ (pokud indexujeme od 1),
se zastavím na indexu $x+1$.

\subsection{Largest Degree coloring}
\begin{algorithm}
\caption{Gredy coloring} %todo citace
\label{Greedy coloring}
\begin{algorithmic}
\Input Neorientovaný graf $G = (V, E)$
\Output funkce obarvení grafu $f: 	V \rightarrow B$
\State $d[1 \dots |V|] \leftarrow []$
\ForAll{$i \in <1, |V|>$}
	\State $d[i] \leftarrow \text{počet sousedících uzlů s }i$
\EndFor
\State $\text{SortedN} \leftarrow \text{Sestupně seřazené uzly podle stupně (d)}$
\ForAll{$v \in \text{SortedN}$}
	\State $S \leftarrow \{ \text{barvy všech obarvených sousedů uzlu v} \}$
	\State $f(v) \leftarrow \text{Nejmenší barva, která není v S}$ 
\EndFor
\State \Return $f$
\end{algorithmic}
\end{algorithm}

Ve své podstatě se jedná o Greedy coloring s nenáhodnou permutací.
Pořadí je zde určeno sestupně na základě stupně uzlu (množství sousedních uzlů).
Časová složitost je zde $O((n* \log n) + n + m)$
Spočtení stupňů uzlu zvládneme v $O(n)$ (předpokládáme znalost velikosti seznamů sousedů, což je triviálně implementovatelné).
Seřazení uzlů podle jejich stupně lze provést v $O(n * \log n)$.
Nakonec provedení greedy coloring má opět složitost $(O(n + m)$. 

Algoritmus by měl za cenu vyšší složitosti dosahovat lepších výsledků (menšího celkového počtu barev) než greedy coloring.

\subsection{Incidence degree ordering}
\begin{algorithm}
\caption{Gredy coloring} %todo citace
\label{Greedy coloring}
\begin{algorithmic}
\Input Neorientovaný graf $G = (V, E)$
\Output funkce obarvení grafu $f: 	V \rightarrow B$
\State $d[1 \dots |V|] \leftarrow []$
\ForAll{$i \in <1, |V|>$}
	\State $d[i] \leftarrow \text{počet sousedících uzlů s }i$
\EndFor
\State $cn[1 \dots |V|] \leftarrow [0]$ - počet obarvených sousedních uzlů
\State $r = V$ - neobarvené uzly
\While{$|r| > 0$}
	\State $v \leftarrow -1$ - neexistující uzel
	\ForAll{$x \in r$}
		\If{$v = -1$ or $cv[x] > cv[v]$}
			\State $v \leftarrow x$
		\EndIf
		\If{$cv[x] = cv[v]$ and $d[x] > d[v]$}
			\State $v \leftarrow x$
		\EndIf
	\EndFor
	\State $S \leftarrow \{ \text{barvy všech obarvených sousedů uzlu v} \}$
	\State $f(v) \leftarrow \text{Nejmenší barva, která není v S}$
	\ForAll{$x \in \{\text{Sousední uzly v}\}$}
		\State $cv[x] \leftarrow cv[x] + 1$
	\EndFor
	\State Remove $v$ from $r$
\EndWhile
\State \Return $f$
\end{algorithmic}
\end{algorithm}

V algoritmu popsaném v \cite{ar4} (a v implementaci) se první uzel obarvuje hned po určení počtů sousedících uzlů; tímto se ušetří jeden cyklus smyčky,
na složitost tato optimalizace nemá vliv a pro zjednodušení pseudokódu nebyla uvedena.

Uzly vybíráme nejdříve podle největšího počtu obarvených sousedů, a v případě stejného počtu obarvených sousedů vybereme pro barvení uzel s vyšším celkovým počtem sousedů.

Spočtení stupňů uzlů zvládneme v $O(n)$.
Samotný cyklus provádíme $n$krát, a v každé iteraci vybíráme uzel k obarvení z průměrně $n / 2$ uzlů.
Nejmenší barva všech uzlů je opět vybraná v celkem $O(m)$.
Odstranění obarveného uzlu lze provést v konstantním čase při použití seznamu pro $r$.
Algoritmus má celkovou složitost $O(n^2 + m)$.

\section{Implementace}
\subsection{Generování grafů}
\subsection{Benchmark pro barvení grafů}

\subsection{Experimenty}

\end{document}
