\documentclass[12pt, a4paper]{article}
\usepackage[czech]{babel}
\usepackage[utf8]{inputenc}
\usepackage{amsmath}
\usepackage{amsthm}
\usepackage{amssymb}
\usepackage{enumitem}
\usepackage{algorithm}
\usepackage{algpseudocode}
\usepackage{mathtools}
\PassOptionsToPackage{hyphens}{url}\usepackage{hyperref}

\usepackage{calrsfs}

\DeclareMathAlphabet{\pazocal}{OMS}{zplm}{m}{n}

\newcommand{\Lb}{\pazocal{L}}

%define Input and Output commands in algorithmicx
\algnewcommand\algorithmicinput{\textbf{Vstup:}}
\algnewcommand\Input{\item[\algorithmicinput]}

\algnewcommand\algorithmicoutput{\textbf{Výstup:}}
\algnewcommand\Output{\item[\algorithmicoutput]}

\let\oldemptyset\emptyset
\let\emptyset\varnothing
\let\oldepsilon\epsilon
\let\epsilon\varepsilon
\let\emptystring\varepsilon

\def\CC{{C\nolinebreak[4]\hspace{-.05em}\raisebox{.4ex}{\tiny\bf ++}}}
\def\CS{{\settoheight{\dimen0}{C}C\kern-.05em \resizebox{!}{\dimen0}{\raisebox{\depth}{\#}}}}

\author{Martin Dočekal \texttt{xdocek09}\\Radek Vít \texttt{xvitra00}}
\title{
	\includegraphics[scale=0.6]{FIT_barevne_PANTONE_CZ.pdf}\\
	GAL 2018\\
	Porovnání složitosti algoritmů barvení grafů
}


\begin{document}
\newpage
\maketitle

\newpage

\section{Úvod}
Tato práce si dává za cíl vytvořit program, který bude schopen experimentálně porovnávat vybrané algoritmy pro barvení grafů. Bude umožňovat i automatické generování grafů, jelikož manuální tvorba velkých grafů (třeba i jen stovky uzlů) je problematická.
V druhé části práce představíme zkoumané algoritmy barvení grafů a ukážeme výsledky experimentů.

\section{Barvení grafů}
Barvení grafů je jedním z problémů z teorie grafů. Cílem barvení je přiřazení každému uzlu barvu (číslo) tak, aby žádný sousední uzel neměl stejnou barvu.
Tomuto říkáme barvení uzlů, tato práce se jím zabývá.
Ovšem existuje například i barvení hran či stěn, kterými se tato práce nezabývá (viz. \cite{lit1}.

Nechť $G=(V,E)$ je neorientovaný graf, kde $n = |V|$ a $m = |E|$. Pak obarvením rozumíme funkci:
$$f: 	V \rightarrow B$$
Kde B je množina barev a $\{ u, v \} \in E \Rightarrow f(u) \not= f(v)$ (viz. \cite{lit2}).
Pro zjednodušení uvažujeme množinu uzlů velikosti $n$ za $V = \{ i \in \mathbb{N} | 1 \leq i \leq n \}$

\section{Vybrané algoritmy barvení grafů}
V této části představíme implementované a zkoumané algoritmy barvení uzlů.

Z důvodu velké časové složitosti obarvení grafu nejmenším možným počtem barev používáme
pro barvení grafu heuristiky, které pracují rychle, ale jejich výsledkem je větší než minimální počet barev.
\subsection{Greedy Coloring}
\begin{algorithm}
\caption{Greedy coloring (viz. \cite{lit3})}
\label{Greedy coloring}
\begin{algorithmic}
\Input Neorientovaný graf $G = (V, E)$
\Output funkce obarvení grafu $f: 	V \rightarrow B$
\State Vyber náhodnou permutaci $p(1), \dots, p(n)$ čísel $1 \dots |V|$
\ForAll{$i \in <1, |V|>$}
	\State $v \leftarrow p(i)$
	\State $S \leftarrow \{ \text{barvy všech obarvených sousedů uzlu v} \}$
	\State $f(v) \leftarrow \text{Nejmenší barva, která není v S}$ 
\EndFor
\State \Return $f$
\end{algorithmic}
\end{algorithm}

Algoritmus prochází uzly v předem stanoveném pořadí.
V pseudokódu je uváděna náhodná permutace, ale v naší implementaci používáme pořadí uzlů, které je dáno načtením/vygenerováním daného grafu.
Každý uzel je obarven nejmenší nepoužitou barvou u svých sousedů.
Z této vlastnosti plyne, že obarvení grafu závisí na zvoleném pořadí uzlů.

Časová složitost algoritmu je $O(n+m)$.
Jelikož procházíme všechny uzly (celkem $O(n)$ iterací)
a při hledání použitelné barvy procházíme všechny sousedy daného uzlu a poznačujeme si použité barvy (celkem $O(m)$).

Nalezení minimální nepoužité barvy lze zvládnout v maximálně $x+1$ krocích (kde $x <= m$).
Protože lze použité barvy ukládat do pole flagů (použitá/nepoužitá) o maximální velikosti $n$ (pro každý uzel jedna barva),
kde index do pole odpovídá dané barvě.
Potom toto pole od začátku procházet a zastavit se na první nepoužité barvě.
Má-li daný uzel $x$ sousedů, tak v nejhorším možném případě, kdy sousedi používají všechny barvy $1$ až $x$ (pokud indexujeme od 1),
se zastavím na indexu $x+1$.

\subsection{Largest Degree ordering (LDO)}
\begin{algorithm}
\caption{LDO coloring (viz. \cite{lit4})} %todo citace
\label{Greedy coloring}
\begin{algorithmic}
\Input Neorientovaný graf $G = (V, E)$
\Output funkce obarvení grafu $f: 	V \rightarrow B$
\State $d[1 \dots |V|] \leftarrow []$
\ForAll{$i \in <1, |V|>$}
	\State $d[i] \leftarrow \text{počet sousedících uzlů s }i$
\EndFor
\State $\text{SortedN} \leftarrow \text{Sestupně seřazené uzly podle stupně (d)}$
\ForAll{$v \in \text{SortedN}$}
	\State $S \leftarrow \{ \text{barvy všech obarvených sousedů uzlu v} \}$
	\State $f(v) \leftarrow \text{Nejmenší barva, která není v S}$ 
\EndFor
\State \Return $f$
\end{algorithmic}
\end{algorithm}

Ve své podstatě se jedná o Greedy coloring s nenáhodnou permutací.
Pořadí je zde určeno sestupně na základě stupně uzlu (množství sousedních uzlů).
Časová složitost je zde $O((n* \log n) + n + m)$
Spočtení stupňů uzlu zvládneme v $O(n)$ (předpokládáme znalost velikosti seznamů sousedů v konstantním čase).
Seřazení uzlů podle jejich stupně lze provést v $O(n * \log n)$.
Nakonec provedení greedy coloring má opět složitost $(O(n + m)$. 

Algoritmus by měl za cenu vyšší složitosti dosahovat lepších výsledků (menšího celkového počtu barev) než greedy coloring.

\subsection{Incidence degree ordering (IDO)}
\begin{algorithm}
\caption{IDO coloring (viz. \cite{lit4})} %todo citace
\label{Greedy coloring}
\begin{algorithmic}
\Input Neorientovaný graf $G = (V, E)$
\Output funkce obarvení grafu $f: 	V \rightarrow B$
\State $d[1 \dots |V|] \leftarrow []$
\ForAll{$i \in <1, |V|>$}
	\State $d[i] \leftarrow \text{počet sousedících uzlů s }i$
\EndFor
\State $cn[1 \dots |V|] \leftarrow [0]$ - počet obarvených sousedních uzlů
\State $r = V$ - neobarvené uzly
\While{$|r| > 0$}
	\State $v \leftarrow -1$ - neexistující uzel
	\ForAll{$x \in r$}
		\If{$v = -1$ or $cv[x] > cv[v]$}
			\State $v \leftarrow x$
		\EndIf
		\If{$cv[x] = cv[v]$ and $d[x] > d[v]$}
			\State $v \leftarrow x$
		\EndIf
	\EndFor
	\State $S \leftarrow \{ \text{barvy všech obarvených sousedů uzlu v} \}$
	\State $f(v) \leftarrow \text{Nejmenší barva, která není v S}$
	\ForAll{$x \in \{\text{Sousední uzly v}\}$}
		\State $cv[x] \leftarrow cv[x] + 1$
	\EndFor
	\State Remove $v$ from $r$
\EndWhile
\State \Return $f$
\end{algorithmic}
\end{algorithm}

V algoritmu popsaném v \cite{ar4} (a v implementaci) se první uzel obarvuje hned po určení počtů sousedících uzlů; tímto se ušetří jeden cyklus smyčky,
na složitost tato optimalizace nemá vliv a pro zjednodušení pseudokódu nebyla uvedena.

Uzly vybíráme nejdříve podle největšího počtu obarvených sousedů, a v případě stejného počtu obarvených sousedů vybereme pro barvení uzel s vyšším celkovým počtem sousedů.

Spočtení stupňů uzlů zvládneme v $O(n)$.
Samotný cyklus provádíme $n$krát, a v každé iteraci vybíráme uzel k obarvení z průměrně $n / 2$ uzlů.
Nejmenší barva všech uzlů je opět vybraná v celkem $O(m)$.
Odstranění obarveného uzlu lze provést v konstantním čase při použití seznamu pro $r$.
Algoritmus má celkovou složitost $O(n^2 + m)$.

\section{Implementované nástroje}
Pro vypracování tohoto projektu jsme implementovali dva nástroje;
nástroj pro měření rychlosti barvení grafů a nástroj pro jejich generování.

\subsection{Generování grafů}
Nástroj pro generování grafů se nachází ve složce \verb|data/|. S pomocí příkazu \verb|make generate N=n E=m| lze vygenerovat neorientovaný graf s $n$ uzly a $m$ hranami.
Nástroj nejdříve náhodně vygeneruje graf s $n$ uzly a s nejvíce $m$ hranami. Následně dogeneruje případné chybějící hrany, nejčastěji do některého z prvních uzlů.

\subsection{Benchmark pro barvení grafů}
Nástroj pro barvení grafů se nachází v kořenovém adresáři projektu.
Lze jej spustit jako \verb|./gal2018 input1 [input2 ...] output.csv|.
Po jeho spuštění nejdříve načte všechny vstupní grafy (hrany jsou reprezentované seznamem sousedů),
a následně spustí pro každý z výše popsaných algoritmů pět běhů.
Jako výsledek se použije medián z těchto pěti běhů pro každý algoritmus.
Z výsledků pro každý algoritmus je do výsledného \verb|.csv| souboru zapsán medián z časů běhů (v milisekundách) daného algoritmu a výsledný počet barev.

Program je limitovaný velikostí paměti, jelikož nejdříve načítá všechny testované grafy do paměti.
Tato limitace se projevuje zejména při velkém množství hustých grafů s velikostmi přes deset tisíc uzlů.

\subsection{Experimenty}
Experimenty jsme provedli v několika bězích, testující nárust počtu uzlů, a nárust počtu hran.
Experimenty byly provedeny s grafy generovanými výše popsaným nástrojem.
V každém experimentu byl použitý jeden náhodný graf s uvedenými počty uzlů a hran.

Výsledky experimentů byly ovlivněné způsobem generace náhodných grafů. Pro greedy coloring bylo velmi pravděpodobné, že
pořadí barvených uzlů téměř odpovídalo seřazení podle jejich stupňů. Tento fakt nicméně ovlivnil pouze počet barev, nikoliv časovou složitost.

Grafy použité v jednotlivých experimentech lze stáhnout z Google Drive: \url{https://drive.google.com/open?id=1eLz35A3oYa0jiQXwKZBvl4DR3HC99kbi}.

Prostorová složitost je pro každý z algoritmů stejná, a to lineární (vstup i případná pomocná pole), nebyla tedy předmětem experimentů.

\subsection{Experiment 1: Izolované grafy}
V tomto experimentu jsme porovnávali časy grafů s 0 hranami.
Testovány byly grafy o velikostech 5\,000, 10\,000, 15\,000 a 20\,000 uzlů.
Výsledky ukazují vliv změny počtu uzlů na rychlost algoritmů.
\begin{table}[h!]
  \begin{center}
    \label{tab:table1}
    \begin{tabular}{|l|c|c|c|}
			\hline
      \textbf{Počet uzlů} & \textbf{Greedy coloring} & \textbf{LDO coloring} & \textbf{IDO coloring}\\
      \hline
       5\,000 & 0.015 & 0.081 & 27.494 \\ \hline
      10\,000 & 0.032 & 0.170 & 118.215 \\ \hline
      15\,000 & 0.047 & 0.244 & 267.494 \\ \hline
      20\,000 & 0.062 & 0.482 & 478.881 \\ \hline
    \end{tabular}
    \caption{Experiment 1: časy [ms]}
  \end{center}
\end{table}

Pro greedy coloring rostl s každým přírustkem 5\,000 výskedný čas o cca 0.015 milisekund, což odpovídá předpokládanému lineárnímu růstu.
Pro LDO coloring růst času odpovídá linearitmické složitosti, t.j. jen o málo rychleji než lineárně.
Pro IDO odpovídá růst časů kvadratickému růstu podle počtu uzlů.

\subsection{Experiment 2: Změna počtu uzlů v grafech se stejným počtem hran}
V tomto experimentu jsme porovnávali časy grafů s 200\,000 hranami.
Testovány byly grafy o velikostech 4\,000, 6\,000, 8\,000 a 10\,000 uzlů.
Výsledky ukazují vliv změny počtu uzlů na rychlost algoritmů.
Zároveň testujeme algoritmy pro řídké grafy (200\,000 uzlů je cca desetina možných uzlů grafu s 2\,000 uzly).
%todo výsledky
\begin{table}[h!]
  \begin{center}
    \label{tab:table1}
    \begin{tabular}{|l|c|c|c|}
			\hline
      \textbf{Počet uzlů} & \textbf{Greedy coloring} & \textbf{LDO coloring} & \textbf{IDO coloring}\\
      \hline
       4\,000 & 1.170 & 2.651 & 119 \\ \hline
       6\,000 & 1.334 & 2.868 & 166 \\ \hline
       8\,000 & 1.835 & 3.175 & 282 \\ \hline
      10\,000 & 1.904 & 4.030 & 410 \\ \hline
    \end{tabular}
    \caption{Experiment 2: časy [ms]}
  \end{center}
\end{table}

\begin{table}[h!]
  \begin{center}
    \label{tab:table1}
    \begin{tabular}{|l|c|c|c|}
			\hline
      \textbf{Počet uzlů} & \textbf{Greedy coloring} & \textbf{LDO coloring} & \textbf{IDO coloring}\\
      \hline
       4\,000 & 42 & 41 & 41 \\ \hline
       6\,000 & 31 & 30 & 30 \\ \hline
       8\,000 & 24 & 24 & 24 \\ \hline
      10\,000 & 21 & 20 & 20 \\ \hline
    \end{tabular}
    \caption{Experiment 2: počet barev}
  \end{center}
\end{table}

\subsection{Experiment 3: Změna počtu hran v hustých grafech}
Poslední experiment testoval husté grafy s 10\,000 uzly s počty hran mezi 10\,000\,000 a 49\,995\,000 (plně propojený graf).

\begin{table}[h!]
  \begin{center}
    \label{tab:table1}
    \begin{tabular}{|l|c|c|c|}
			\hline
      \textbf{Počet hran} & \textbf{Greedy coloring} & \textbf{LDO coloring} & \textbf{IDO coloring}\\
      \hline
       10\,000\,000 & 77  & 110 & 439 \\ \hline
       20\,000\,000 & 146 & 209 & 502 \\ \hline
       30\,000\,000 & 208 & 284 & 568 \\ \hline
       40\,000\,000 & 260 & 317 & 626 \\ \hline
       49\,995\,000 & 321 & 334 & 746 \\ \hline
    \end{tabular}
    \caption{Experiment 3: časy [ms]}
  \end{center}
\end{table}


\begin{table}[h!]
  \begin{center}
    \label{tab:table1}
    \begin{tabular}{|l|c|c|c|}
			\hline
      \textbf{Počet hran} & \textbf{Greedy coloring} & \textbf{LDO coloring} & \textbf{IDO coloring}\\
      \hline
       10\,000\,000 & 720  & 716 & 716 \\ \hline
       20\,000\,000 & 1595 & 1591 & 1591 \\ \hline
       30\,000\,000 & 2780 & 2773 & 2773 \\ \hline
       40\,000\,000 & 4587 & 4582 & 4582 \\ \hline
       49\,995\,000 & 10000 & 10000 & 10000 \\ \hline
    \end{tabular}
    \caption{Experiment 3: počet barev}
  \end{center}
\end{table}

Všechny uvedené algoritmy měly podle teoretické složitosti vůči počtům hran růst lineárně.
Čas greedy algoritmu i IDO rostl téměř lineárně podle předpokladu.
LDO zpomaloval pomaleji až při velmi velké hustotě hran, pro ostatní grafy také zpomaloval lineárně podle počtu hran.

\section{Závěr}
Implementovali jsme některé algoritmy pro barvení grafů,
a teoreticky a experimentálně jsme porovnali jejich časovou složitost.

Všechny uvedené algoritmy mají vůči počtu hran lineární časovou složitost, a experimentální výsledky tomuto předpokladu odpovídají.
Experimenty také odpovídají odvozeným časovým složitostem jednotlivých algoritmů vůči počtu uzlů.

Greedy coloring je (z uvedených algoritmů) algoritmus s nejlepší časovou složitostí a pro testované grafy srovnatelnými výsledky jako ostatní algoritmy.
V žádném testovaném grafu se nevyskytnul graf, který by měl pro LDO horší výsledky než IDO.
IDO tak za vyšší složitost nepřinesl lepší výsledky.

\renewcommand\refname{Literatura}
\bibliographystyle{plain}
\begin{flushleft}
\bibliography{report}
\end{flushleft}

\end{document}
